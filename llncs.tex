% This is LLNCS.DEM the demonstration file of
% the LaTeX macro package from Springer-Verlag
% for Lecture Notes in Computer Science,
% version 2.4 for LaTeX2e as of 16. April 2010
%
\documentclass{llncs}
%
\usepackage{makeidx}  % allows for indexgeneration
%
\begin{document}
%
\frontmatter          % for the preliminaries
%
\pagestyle{headings}  % switches on printing of running heads
\addtocmark{Current Status and Trends for Mobile Clouds} % additional mark in the TOC

%
\tableofcontents
%
\mainmatter              % start of the contributions
%
\title{Current Status and Trends for Mobile Clouds}
%
\titlerunning{Current Status and Trends for Mobile Clouds}  % abbreviated title (for running head)
%                                     also used for the TOC unless
%                                     \toctitle is used
%
\author{Martin Keiblinger, 0825118}
%
\authorrunning{Martin Keiblinger et al.} % abbreviated author list (for running head)
%
%%%% list of authors for the TOC (use if author list has to be modified)
\tocauthor{Martin Keiblinger}
%
\institute{TU Wien, Karlsplatz 13, 1040 Wien, Austria,\\
\email{e0825118@student.tuwien.ac.at}}

\maketitle
%
% Modify the bibliography environment to call for the author-year
% system. This is done normally with the citeauthoryear option
% for a particular contribution.
\makeatletter
\renewenvironment{thebibliography}[1]
     {\section*{\refname}
      \small
      \list{}%
           {\settowidth\labelwidth{}%
            \leftmargin\parindent
            \itemindent=-\parindent
            \labelsep=\z@
            \if@openbib
              \advance\leftmargin\bibindent
              \itemindent -\bibindent
              \listparindent \itemindent
              \parsep \z@
            \fi
            \usecounter{enumiv}%
            \let\p@enumiv\@empty
            \renewcommand\theenumiv{}}%
      \if@openbib
        \renewcommand\newblock{\par}%
      \else
        \renewcommand\newblock{\hskip .11em \@plus.33em \@minus.07em}%
      \fi
      \sloppy\clubpenalty4000\widowpenalty4000%
      \sfcode`\.=\@m}
     {\def\@noitemerr
       {\@latex@warning{Empty `thebibliography' environment}}%
      \endlist}
      \def\@cite#1{#1}%
      \def\@lbibitem[#1]#2{\item[]\if@filesw
        {\def\protect##1{\string ##1\space}\immediate
      \write\@auxout{\string\bibcite{#2}{#1}}}\fi\ignorespaces}
\makeatother
%
\begin{abstract}
Due to the increasing spreading of smart phones a new market is emerging. As a smart phone is a multi sensor device with at least one communication modul available and processors which get more powerful every year, it makes sense to move needed calculations for different services to the phones itself. This paper summarizes the different approaches of distributing calculation tasks and technical constraints. Due to the nature of smart phones they only can build unreliable ad-hoc networks. \dots
\keywords{graph transformations, convex geometry, lattice computations,
convex polygons, triangulations, discrete geometry}
\end{abstract}
%
\section{Introduction}
%

\subsection{Clouds}

\subsection{Mobile clouds}

%
\section{Task distribution models}

\subsection{Client/Server modell}

\subsection{Virtualization}

\subsection{Mobile agents}
%

%
\section{Technical constraints and solutions}

\subsection{Energy consumption}
What protocols should be used? What wifi technologies are energy efficient? 

\subsection{Implementation issues}
What constraints are set by operating systems? How could a mesh network build with iOS devices? With Android devices? Is it possible to connect Android and iOS-Devices?
%

\section{Case studies in mobile clouds}
Samsung project. Other projects?

\paragraph{Notes and Comments.}
The first results on subharmonics were
obtained by Rabinowitz in \cite{2rab}, who showed the existence of
infinitely many subharmonics both in the subquadratic and superquadratic
case, with suitable growth conditions on $H'$. Again the duality
approach enabled Clarke and Ekeland in \cite{2clar:eke:2} to treat the
same problem in the convex-subquadratic case, with growth conditions on
$H$ only.

Recently, Michalek and Tarantello (see Michalek, R., Tarantello, G.
\cite{2mich:tar} and Tarantello, G. \cite{2tar}) have obtained lower
bound on the number of subharmonics of period $kT$, based on symmetry
considerations and on pinching estimates, as in Sect.~5.2 of this
article.

%
% ---- Bibliography ----
%
\begin{thebibliography}{}
%
\bibitem[1980]{2clar:eke}
Clarke, F., Ekeland, I.:
Nonlinear oscillations and
boundary-value problems for Hamiltonian systems.
Arch. Rat. Mech. Anal. 78, 315--333 (1982)

\bibitem[1981]{2clar:eke:2}
Clarke, F., Ekeland, I.:
Solutions p\'{e}riodiques, du
p\'{e}riode donn\'{e}e, des \'{e}quations hamiltoniennes.
Note CRAS Paris 287, 1013--1015 (1978)

\bibitem[1982]{2mich:tar}
Michalek, R., Tarantello, G.:
Subharmonic solutions with prescribed minimal
period for nonautonomous Hamiltonian systems.
J. Diff. Eq. 72, 28--55 (1988)

\bibitem[1983]{2tar}
Tarantello, G.:
Subharmonic solutions for Hamiltonian
systems via a $\bbbz_{p}$ pseudoindex theory.
Annali di Matematica Pura (to appear)

\bibitem[1985]{2rab}
Rabinowitz, P.:
On subharmonic solutions of a Hamiltonian system.
Comm. Pure Appl. Math. 33, 609--633 (1980)

\end{thebibliography}
\clearpage
\addtocmark[2]{Author Index} % additional numbered TOC entry
\renewcommand{\indexname}{Author Index}
\printindex
\clearpage
\addtocmark[2]{Subject Index} % additional numbered TOC entry
\markboth{Subject Index}{Subject Index}
\renewcommand{\indexname}{Subject Index}
\input{subjidx.ind}
\end{document}
